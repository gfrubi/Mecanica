\begin{thebibliography}{99}
\bibitem{mit2003}
\bibitem{mit2009} MIT OCW, Physics I, Classical Mechanics,  \url{http://goo.gl/jJSb3}\footnote{These MIT OpenCourseWare course materials have been translated into Spanish by DR. The MIT faculty authors, MIT, or MIT OpenCourseWare have not reviewed or approved these translations, and MIT and MIT OpenCourseWare makes no representations or warranties of any kind concerning the translated materials, express or implied, including, without limitation, warranties of merchantability, fitness for a particular purpose, non-infringement, or the absence of errors, whether or not discoverable. MIT OpenCourseWare bears no responsibility for any inaccuracies in translation. Any inaccuracies or other defects contained in this material, due to inaccuracies in language translation, are the sole responsibility of DR and not MIT OpenCourseWare.}
\bibitem{Kleppner} D. Kleppner, R.J Kolenkow, An Introduction To Mechanics, McGrawHill
\bibitem{Morin} David Morin, Introduction to Classical Mechanics, Cambridge
\bibitem{Young} Young, Hugh D., Roger A. Freedman, and A. Lewis Ford.
\bibitem{Sears} Sears and Zemansky's University Physics: with Modern Physics. 12th ed. San Francisco, CA: Addison-Wesley, 2007. ISBN: 9780805321876.
\bibitem{MI} Ruth W. Chabay, Bruce A. Sherwood, Matter \& Interactions, Volume I: Moder Mechanics, John Wiley \& Sons, Inc, 3rd ed., 2011. ISBN: 9780470503478
\bibitem{gabriel} Teoría y Problemas de Física I, Gabriel Jaime Pérez Lince, disponible en la fotocopiadora de Ingeniería (Con muchos problemas resueltos!)
\bibitem{rs} J.H.Field, Space-Time Exchange Invariance: Special Relativity as a Symmetry Principle, Am. J. Phys. 69 (2001) 569-575. DOI:	10.1119/1.1344165, arXiv:physics/0012011 [physics.class-ph]
\bibitem{rse} A. R. Lee,  T.M. Kalotas, Lorentz transformations from the first postulate,
  Am. J. Phys. Vol 43. No. 5, May 1975.
\bibitem{rsf} Jean-Marc Lévy-Leblond, One more derivation of the Lorentz transformation, American Journal of Physics Vol. 44, No. 3, March 1976
 \bibtem{rsg} Alan Macdonald, Still the ``World's Fastest Derivation of the Lorentz Transformation'',
  	arXiv:physics/0606046 [physics.class-ph]

\end{thebibliography}
%%% Local Variables: 
%%% mode: latex
%%% TeX-master: "mecanica"
%%% End:

